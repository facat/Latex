% a4paper - A4纸  11pt -字体 twoside -双面 openany -新章节可在偶数页开始
\documentclass[a4paper,11pt,twoside,openany]{article}
%\usepackage{CJK}
\usepackage{geometry}
\usepackage{amsmath}%调用数学包
\usepackage{ccmap}                      % 使pdfLatex生成的文件支持复制等
\usepackage{times}                       % 使用 Times New Roman 字体
\geometry{left=2.5cm,right=2.5cm,top=2.5cm,bottom=2.5cm}

\usepackage{indentfirst}%设置缩进
\usepackage{anysize}%设置16k
\papersize{26cm}{18.4cm}%设置16k
\marginsize{2.5cm}{2.1cm}{2cm}{2cm} %设置16k
\usepackage{fontspec}%使用xetex
\setmainfont[BoldFont=黑体]{宋体}               % 使用系统默认字体
\XeTeXlinebreaklocale "zh"                      % 针对中文进行断行
\XeTeXlinebreakskip = 0pt plus 1pt minus 0.1pt  % 给予TeX断行一定自由度
\linespread{1.5}                                % 1.5倍行距
\setcounter{page}{0}
\newcommand{\upcite}[1]{$^{\mbox{\scriptsize \cite{#1}}}$}%上标引用
\begin{document}

\tableofcontents
%\pagestyle{empty}     %第二頁以後頁碼空白
\thispagestyle{empty}    %首頁頁碼空白  
\author{Your Name}
\title{Test Document}
\maketitle
\today %当日时间
\begin{abstract}
这是摘要
\end{abstract}
\vfill %用空白填充,使内容充满整页
\textbf{关键词:} 学位论文,\LaTeX\ 模版,科技排版

%\setlength{\parindent}{2em}%设置缩进

This is a test document
\newline %换行
%\begin{CJK*}{GBK}{song}

测试一下我的数学公式\[\int_a^b f(x)dx\]

到这里
\footnote{一段的脚注}
继续写。
\par
引用一下\cite{zhangshan}
%\newcommand{\upcite}[1]{$^{\mbox{\scriptsize \cite{#1}}}$}
上标引用\upcite{zhangshan}
%\end{CJK*}
\begin{thebibliography}{9}
\bibitem{zhangshan}
Leslie Lamport,
\emph{\LaTeX: A Document Preparation System}.
Addison Wesley, Massachusetts,
2nd Edition,
1994.
\end{thebibliography}

\end{document}

